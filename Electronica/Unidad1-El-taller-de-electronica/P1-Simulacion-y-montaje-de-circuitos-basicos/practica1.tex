\documentclass[a4paper,11pt]{article}

% Importar preamble de prácticas
% =============================================================================
% PREAMBLE PARA DOCUMENTOS DE PRÁCTICAS
% =============================================================================
% Este preamble está diseñado para hojas de prácticas y ejercicios
% Incluye cajas personalizadas, tablas, y formato para trabajo de laboratorio

% --- Paquetes básicos ---
\usepackage[spanish]{babel}
\usepackage[utf8]{inputenc}
\usepackage{amsmath, amssymb}
\usepackage{graphicx}
\usepackage{float}
\usepackage{multicol}
\usepackage{array}

% --- Geometría de página ---
\usepackage{geometry}
\geometry{left=2.5cm,right=2.5cm,top=2.5cm,bottom=2.5cm}

% --- Cajas y diseño ---
\usepackage{tcolorbox}
\usepackage{booktabs}

% --- Estilo para títulos ---
\usepackage{titlesec}
\titleformat{\section}{\bfseries\large}{\thesection}{1em}{}
\titleformat{\subsection}{\bfseries}{\thesubsection}{1em}{}

% --- Cabecera y pie de página ---
% NOTA: Personaliza \rhead en cada documento según la asignatura
\usepackage{fancyhdr}
\pagestyle{fancy}
\fancyhf{}
\lhead{Prácticas}
% \rhead se define en cada documento individual
\cfoot{\thepage}

% --- Estilo de caja para ejercicios (blanco y negro) ---
\tcbset{
  mybox/.style={
    colback=white,           % fondo blanco
    colframe=black,          % borde negro
    fonttitle=\bfseries,
    boxrule=0.8pt,           % grosor del borde
    arc=3mm,                 % esquinas redondeadas
    enhanced,
    sharp corners=downhill,  % estilo limpio
  }
}


% Personalización de la asignatura
\rhead{Electrónica}

% -------------------------
\begin{document}

\begin{center}
    \Huge \textbf{Práctica 1: Simulación y montaje de circuitos básicos.} \\[0.5cm]
\end{center}

\noindent \large Fecha: \\
\large Nombre alumno:\\

\begin{tcolorbox}[mybox, title=Ejercicio 1]
Utiliza Tinkercad como software de simulación electrónica y realiza el diseño de 3 resistencias en serie como se muestra en la figura. Anota los siguientes resultados en la tabla:
\end{tcolorbox}

\begin{figure}[H]
    \centering
    \includegraphics[width=0.7\textwidth]{img/resistencias_serie.png}
    \label{fig:resistencias_serie}
\end{figure}

\begin{table}[h!]
    \Large
    \centering
    \begin{tabular}{ccccc}
        \toprule
        \textbf{Tensión de la fuente} & \textbf{I} & \textbf{$V_1$} & \textbf{$V_2$} & \textbf{$V_3$} \\
        \midrule
        5 V  &   &   &   &   \\
        \midrule
        9 V  &   &   &   &   \\
        \midrule
        12 V &   &   &   & \\
        \bottomrule
    \end{tabular}
    \label{tab:medidas_circuito_serie}
\end{table}

\begin{tcolorbox}[mybox, title=Ejercicio 2]
Utiliza Tinkercad como software de simulación electrónica y realiza el diseño de 3 resistencias en paralelo y anota lo siguientes resultados:
\end{tcolorbox}

\begin{figure}[H]
    \centering
    \includegraphics[width=0.7\textwidth]{img/resistencias_paralelo.png}
    \label{fig:resistencias_paralelo}
\end{figure}

\begin{table}[h!]
    \Large
    \centering
    \begin{tabular}{cccccc}
        \toprule
        \textbf{Tensión de la fuente} & \textbf{$I_t$} & \textbf{$I_1$} & \textbf{$I_2$} & \textbf{$I_3$} & $V_1, V_2, V_3$ \\
        \midrule
        5 V  &   &   &   &  & \\
        \midrule
        9 V  &   &   &   &  & \\
        \midrule
        12 V &   &   &   & & \\
        \bottomrule
    \end{tabular}
    \label{tab:medidas_circuito_paralelo}
\end{table}


\begin{tcolorbox}[mybox, title=Ejercicio 3]
Sobre una placa de prototipos, conecta en serie las tres resistencias del Ejercicio 1. Conecta el circuito a una fuente de alimentación de tensión variable y, utilizando un polímetro, anota en la tabla los resultados obtenidos en función del valor de tensión ajustado en la fuente.
\end{tcolorbox}

\begin{table}[h!]
    \Large
    \centering
    \begin{tabular}{ccccc}
        \toprule
        \textbf{Tensión de la fuente} & \textbf{I} & \textbf{$V_1$} & \textbf{$V_2$} & \textbf{$V_3$} \\
        \midrule
        5 V  &   &   &   &   \\
        \midrule
        9 V  &   &   &   &   \\
        \midrule
        12 V &   &   &   & \\
        \bottomrule
    \end{tabular}
    \label{tab:medidas_montaje_serie}
\end{table}

\begin{tcolorbox}[mybox, title=Ejercicio 4]
Sobre una placa de prototipos monta tres resistencias en paralelo (Ejercicio 2) y, siguiendo el mismo procedimiento anterior anotad las medidas de corrientes y tensiones correspondientes y rellenad la tabla:
\end{tcolorbox}

\begin{table}[h!]
    \Large
    \centering
    \begin{tabular}{cccccc}
        \toprule
        \textbf{Tensión de la fuente} & \textbf{$I_T$} & \textbf{$I_1$} & \textbf{$I_2$} & \textbf{$I_3$} & $V_1, V_2, V_3$ \\
        \midrule
        5 V  &   &   &   &  & \\
        \midrule
        9 V  &   &   &   &  & \\
        \midrule
        12 V &   &   &   & & \\
        \bottomrule
    \end{tabular}
    \label{tab:medidas_montaje_paralelo}
\end{table}

\end{document}
