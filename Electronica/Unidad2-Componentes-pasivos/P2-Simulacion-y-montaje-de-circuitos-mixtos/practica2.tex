\documentclass[a4paper,11pt]{article}

% Importar preamble de prácticas
% =============================================================================
% PREAMBLE PARA DOCUMENTOS DE PRÁCTICAS
% =============================================================================
% Este preamble está diseñado para hojas de prácticas y ejercicios
% Incluye cajas personalizadas, tablas, y formato para trabajo de laboratorio

% --- Paquetes básicos ---
\usepackage[spanish]{babel}
\usepackage[utf8]{inputenc}
\usepackage{amsmath, amssymb}
\usepackage{graphicx}
\usepackage{float}
\usepackage{multicol}
\usepackage{array}

% --- Geometría de página ---
\usepackage{geometry}
\geometry{left=2.5cm,right=2.5cm,top=2.5cm,bottom=2.5cm}

% --- Cajas y diseño ---
\usepackage{tcolorbox}
\usepackage{booktabs}

% --- Estilo para títulos ---
\usepackage{titlesec}
\titleformat{\section}{\bfseries\large}{\thesection}{1em}{}
\titleformat{\subsection}{\bfseries}{\thesubsection}{1em}{}

% --- Cabecera y pie de página ---
% NOTA: Personaliza \rhead en cada documento según la asignatura
\usepackage{fancyhdr}
\pagestyle{fancy}
\fancyhf{}
\lhead{Prácticas}
% \rhead se define en cada documento individual
\cfoot{\thepage}

% --- Estilo de caja para ejercicios (blanco y negro) ---
\tcbset{
  mybox/.style={
    colback=white,           % fondo blanco
    colframe=black,          % borde negro
    fonttitle=\bfseries,
    boxrule=0.8pt,           % grosor del borde
    arc=3mm,                 % esquinas redondeadas
    enhanced,
    sharp corners=downhill,  % estilo limpio
  }
}


\begin{document}

\begin{center}
    \Huge \textbf{Práctica 2: Simulación y montaje de circuitos mixtos.} \\[0.5cm]
\end{center}

\noindent \large Fecha: \\
\large Nombre alumno:\\

\begin{tcolorbox}[mybox, title=Ejercicio 1]
Utiliza Tinkercad como software de simulación electrónica y realiza el diseño del circuito mixto propuesto a continuación y anota los resultados en la tabla:
\end{tcolorbox}

\begin{figure}[H]
    \centering
    \includegraphics[width=0.9\textwidth]{img/circuito mixto.png}
    \label{fig:resistencias_serie}
\end{figure}


\begin{table}[h!]
    \Large
    \centering
    \renewcommand{\arraystretch}{1.6}
    \begin{tabularx}{\textwidth}{|>{\centering\arraybackslash}X|
                                 >{\centering\arraybackslash}X|
                                 >{\centering\arraybackslash}X|
                                 >{\centering\arraybackslash}X|
                                 >{\centering\arraybackslash}X|
                                 >{\centering\arraybackslash}X|
                                 >{\centering\arraybackslash}X|}
        \hline
        \multicolumn{3}{|c|}{\textbf{Tensión}} & 
        \multicolumn{4}{c|}{\textbf{Intensidad}} \\
        \hline
        \textbf{$V_1$} & \textbf{$V_2$} & \textbf{$V_3$} & \textbf{$I_T$} & \textbf{$I_1$} & \textbf{$I_2$} & \textbf{$I_3$} \\
        \hline
        & & & & & & \\
        \hline
    \end{tabularx}
    \label{tab:tension_intensidad}
\end{table}

\begin{table}[h!]
    \Large
    \centering
    \renewcommand{\arraystretch}{1.6}
    \begin{tabularx}{\textwidth}{|>{\centering\arraybackslash}X|
                                 >{\centering\arraybackslash}X|}
        \hline
        \textbf{Resistencia total} & \textbf{Resistencia circuito paralelo} \\
        \hline
         &  \\
        \hline
    \end{tabularx}
    \label{tab:resistencias_equivalentes}
\end{table}



\begin{tcolorbox}[mybox, title=Ejercicio 2]
Monta el circuito del Ejercicio 1 en la placa protoboard. Utiliza un multímetro para realizar las medidas correspondientes para rellenar la tabla y comprar los resultados.
\end{tcolorbox}


\begin{table}[h!]
    \Large
    \centering
    \renewcommand{\arraystretch}{1.6}
    \begin{tabularx}{\textwidth}{|>{\centering\arraybackslash}X|
                                 >{\centering\arraybackslash}X|
                                 >{\centering\arraybackslash}X|
                                 >{\centering\arraybackslash}X|
                                 >{\centering\arraybackslash}X|
                                 >{\centering\arraybackslash}X|
                                 >{\centering\arraybackslash}X|}
        \hline
        \multicolumn{3}{|c|}{\textbf{Tensión}} & 
        \multicolumn{4}{c|}{\textbf{Intensidad}} \\
        \hline
        \textbf{$V_1$} & \textbf{$V_2$} & \textbf{$V_3$} & \textbf{$I_T$} & \textbf{$I_1$} & \textbf{$I_2$} & \textbf{$I_3$} \\
        \hline
        & & & & & & \\
        \hline
    \end{tabularx}
    \label{tab:tension_intensidad}
\end{table}


\begin{table}[h!]
    \Large
    \centering
    \renewcommand{\arraystretch}{1.6}
    \begin{tabularx}{\textwidth}{|>{\centering\arraybackslash}X|
                                 >{\centering\arraybackslash}X|}
        \hline
        \textbf{Resistencia total} & \textbf{Resistencia circuito paralelo} \\
        \hline
         &  \\
        \hline
    \end{tabularx}
    \label{tab:resistencias_equivalentes}
\end{table}

\end{document}
