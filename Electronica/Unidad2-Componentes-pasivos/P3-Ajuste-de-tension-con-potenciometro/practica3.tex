\documentclass[a4paper,11pt]{article}

% Importar preamble de prácticas
% =============================================================================
% PREAMBLE PARA DOCUMENTOS DE PRÁCTICAS
% =============================================================================
% Este preamble está diseñado para hojas de prácticas y ejercicios
% Incluye cajas personalizadas, tablas, y formato para trabajo de laboratorio

% --- Paquetes básicos ---
\usepackage[spanish]{babel}
\usepackage[utf8]{inputenc}
\usepackage{amsmath, amssymb}
\usepackage{graphicx}
\usepackage{float}
\usepackage{multicol}
\usepackage{array}

% --- Geometría de página ---
\usepackage{geometry}
\geometry{left=2.5cm,right=2.5cm,top=2.5cm,bottom=2.5cm}

% --- Cajas y diseño ---
\usepackage{tcolorbox}
\usepackage{booktabs}

% --- Estilo para títulos ---
\usepackage{titlesec}
\titleformat{\section}{\bfseries\large}{\thesection}{1em}{}
\titleformat{\subsection}{\bfseries}{\thesubsection}{1em}{}

% --- Cabecera y pie de página ---
% NOTA: Personaliza \rhead en cada documento según la asignatura
\usepackage{fancyhdr}
\pagestyle{fancy}
\fancyhf{}
\lhead{Prácticas}
% \rhead se define en cada documento individual
\cfoot{\thepage}

% --- Estilo de caja para ejercicios (blanco y negro) ---
\tcbset{
  mybox/.style={
    colback=white,           % fondo blanco
    colframe=black,          % borde negro
    fonttitle=\bfseries,
    boxrule=0.8pt,           % grosor del borde
    arc=3mm,                 % esquinas redondeadas
    enhanced,
    sharp corners=downhill,  % estilo limpio
  }
}


\begin{document}

\begin{center}
    \Huge \textbf{Práctica 3: Ajuste de tensión con potenciómetro.} \\[0.5cm]
\end{center}

\noindent \large Fecha: \\
\large Nombre alumno:\\

\begin{tcolorbox}[mybox, title=Ejercicio 1]
Realiza el montaje del siguiente circuito divisor de tensión para comprobar el funcionamiento del potenciómetro para realizar un ajuste de tensión a la salida deseada. Anota el valor de tensión obtenido en la tabla:
\end{tcolorbox}

\begin{figure}[H]
    \centering
    \includegraphics[width=0.7\textwidth]{img/potenciometro.png}
    \label{fig:resistencias_serie}
\end{figure}

\begin{table}[h!]
    \Large
    \centering
    \renewcommand{\arraystretch}{1.6}
    \begin{tabularx}{\textwidth}{|>{\centering\arraybackslash}X|
                                 >{\centering\arraybackslash}X| >{\centering\arraybackslash}X|}
        \hline
        \textbf{$V_{R=1k\Omega}$} & \textbf{$V_{R=5k\Omega}$} & $I_T$ \\
        \hline
         & & \\
         \hline
    \end{tabularx}
    \label{tab:potenciometro_tension}
\end{table}

\newpage

\begin{tcolorbox}[mybox, title=Ejercicio 2]
Monta el circuito del ejercicio 1 en la placa protooboard. Varía el valor del potenciómetro (entre el máximo, el mínimo y dos valores intermedios) y comprueba el valor de tensión de salida.
\end{tcolorbox}

\begin{table}[h!]
    \Large
    \centering
    \renewcommand{\arraystretch}{1.6}
    \begin{tabularx}{\textwidth}{|>{\centering\arraybackslash}X|
                                 >{\centering\arraybackslash}X|}
        \hline
        \textbf{Posición del potenciómetro $(\Omega)$} & \textbf{Tensión de salida (V)} \\
        \hline
         &  \\
        \hline
         &  \\
        \hline
        &  \\
        \hline 
         &  \\
        \hline
    \end{tabularx}
    \label{tab:potenciometro_tension}
\end{table}

\begin{tcolorbox}[mybox, title=Ejercicio 3]
Cambia el potenciómetro por otro de $1k\Omega$. ¿Qué ocurre en este caso con el valor máximo de tensión en el punto de medida?
\end{tcolorbox}

\begin{table}[h!]
    \Large
    \centering
    \renewcommand{\arraystretch}{1.6}
    \begin{tabularx}{\textwidth}{|>{\centering\arraybackslash}X|
                                 >{\centering\arraybackslash}X|}
        \hline
        \textbf{Posición del potenciómetro} & \textbf{Tensión de salida (V)} \\
        \hline
         &  \\
        \hline
         &  \\
        \hline
    \end{tabularx}
    \label{tab:potenciometro_tension}
\end{table}




\end{document}
