\documentclass[a4paper,11pt]{article}

% Importar preamble de prácticas
% =============================================================================
% PREAMBLE PARA DOCUMENTOS DE PRÁCTICAS
% =============================================================================
% Este preamble está diseñado para hojas de prácticas y ejercicios
% Incluye cajas personalizadas, tablas, y formato para trabajo de laboratorio

% --- Paquetes básicos ---
\usepackage[spanish]{babel}
\usepackage[utf8]{inputenc}
\usepackage{amsmath, amssymb}
\usepackage{graphicx}
\usepackage{float}
\usepackage{multicol}
\usepackage{array}

% --- Geometría de página ---
\usepackage{geometry}
\geometry{left=2.5cm,right=2.5cm,top=2.5cm,bottom=2.5cm}

% --- Cajas y diseño ---
\usepackage{tcolorbox}
\usepackage{booktabs}

% --- Estilo para títulos ---
\usepackage{titlesec}
\titleformat{\section}{\bfseries\large}{\thesection}{1em}{}
\titleformat{\subsection}{\bfseries}{\thesubsection}{1em}{}

% --- Cabecera y pie de página ---
% NOTA: Personaliza \rhead en cada documento según la asignatura
\usepackage{fancyhdr}
\pagestyle{fancy}
\fancyhf{}
\lhead{Prácticas}
% \rhead se define en cada documento individual
\cfoot{\thepage}

% --- Estilo de caja para ejercicios (blanco y negro) ---
\tcbset{
  mybox/.style={
    colback=white,           % fondo blanco
    colframe=black,          % borde negro
    fonttitle=\bfseries,
    boxrule=0.8pt,           % grosor del borde
    arc=3mm,                 % esquinas redondeadas
    enhanced,
    sharp corners=downhill,  % estilo limpio
  }
}




\begin{document}

\begin{center}
    \Huge \textbf{Práctica 4. Fabricación de un \textit{display} de 7 segmentos.} \\[0.5cm]
\end{center}

\noindent \large Fecha: \\
\large Nombre alumno:\\

\begin{tcolorbox}[mybox, title= Objetivo de la práctica]
El objetivo de esta práctica es fabricar un display de 7 segmentos utilizando una estructura impresa en 3D y que posteriormente se controlará mediante una placa Arduino con la finalidad de construir un reloj funcional. Se dispone de una configuración de 21 diodos LED conectados en cátodo común.
\end{tcolorbox}

\vspace{0.5cm}

\begin{center}
\begin{circuitikz}[scale=0.9, transform shape]

    %%%%% LEDS L1 %%%%%%
    \draw (1,11) to (1,15) to[resistor, l^={$R = 100 \Omega$}] (-2,15) node[fill=black, circle, inner sep=2pt] (L7) {} node[left]{L1};

    \draw (1,15) to (2,15) to [led] (2,14) to (3,14);
    \draw (1,13) to (2,13) to [led] (2,12) to (3,12);
    \draw (1,11) to (2,11) to [led] (2,10) to (3,10);

    \draw (3,16) to (3,0);

    %%%%% LEDS L2 %%%%%%
    \draw (3.5,17) to (10,17) to[resistor, l^={$R = 100 \Omega$}] (13,17) node[fill=black, circle, inner sep=2pt] (L7) {} node[right]{L2};
    
    \draw (3.5,17) to[led] (3.5,16);
    \draw (5.5,17) to[led] (5.5,16);
    \draw (7.5,17) to[led] (7.5,16);

    \draw (3,16) to (7,16);

    \draw (3,16) to (8,16) to (8,0);

    %%%%% LEDS L3 %%%%%%
    \draw (10,11) to (10,15) to[resistor, l^={$R = 100 \Omega$}] (13,15) node[fill=black, circle, inner sep=2pt] (L7) {} node[right]{L3};
    
    \draw (10,15) to (9,15) to [led] (9,14) to (8,14);
    \draw (10,13) to (9,13) to [led] (9,12) to (8,12);
    \draw (10,11) to (9,11) to [led] (9,10) to (8,10);


    %%%%% LEDS L4 %%%%%%
    \draw (5.5,9) to[resistor, l^={$R = 100 \Omega$}] (5.5,11) node[fill=black, circle, inner sep=2pt] (L7) {} node[right]{L4};
    
    \draw (3.5,9) to (7.5,9);
    
    \draw (3.5,9) to[led] (3.5,8);
    \draw (5.5,9) to[led] (5.5,8);
    \draw (7.5,9) to[led] (7.5,8);

    \draw (3,8) to (8,8);

    %%%%% LEDS L5 %%%%%%
    \draw (10,3) to (10,7) to[resistor, l^={$R = 100 \Omega$}] (13,7) node[fill=black, circle, inner sep=2pt] (L7) {} node[right]{L5};
    
    \draw (10,7) to (9,7) to [led] (9,6) to (8,6);
    \draw (10,5) to (9,5) to [led] (9,4) to (8,4);
    \draw (10,3) to (9,3) to [led] (9,2) to (8,2);

    %%%%%% LEDS L6 %%%%%%
    \draw (5.5,3) node[fill=black, circle, inner sep=2pt] (L6) {} node[left]{L6} to[resistor, l_={$R = 100 \Omega$}, l^={$R = 100 \Omega$}] (5.5,1) -- (7.5,1) to (3.5,1);
    
    \draw (3.5,1) to[led] (3.5,0);
    \draw (5.5,1) to[led] (5.5,0);
    \draw (7.5,1) to[led] (7.5,0);

    \draw (3,0) to (8,0);
    
    %%%%% LEDS L7 %%%%%%
    \draw (-2,7) node[fill=black, circle, inner sep=2pt] (L7) {} node[left]{L7} to[resistor, l_={$R = 100 \Omega$}] (1,7) to (1,3) ;

    \draw (1,7) to (2,7) to [led] (2,6) to (3,6);
    \draw (1,5) to (2,5) to [led] (2,4) to (3,4);
    \draw (1,3) to (2,3) to [led] (2,2) to (3,2);

    %%%%% GND %%%%%%
    \draw (5.5,8) to (5.5,7) node[fill=black, circle, inner sep=2pt] (L7) {} node[left]{GND} ;

\end{circuitikz}
\end{center}


\begin{tcolorbox}[mybox, title=Pasos a seguir para realizar el montaje:]


\begin{enumerate}
    \item Identificar todos los materiales necesarios para el proyecto.
    \item Revisar el circuito esquemático, identificar los materiales necesarios y puntos de conexión.
    \item Conectar todos los diodos en la misma posición (ánodo arriba, cátodo abajo).
    \item Pegar todos los diodos con cola a la estructura.
    \item Comprobar que todos los diodos esté en la posición correcta.
    \item Empezar a soldar las líneas de los ánodos (L1 a L7).
    \item Soldar todos los cátodos juntos.
    \item Conectar cables de salida y numerar adecuadamente todas las líneas.
\end{enumerate}
\end{tcolorbox}

\vspace{1cm}

\begin{tcolorbox}[mybox, title=Ejercicio 1]
Rellena la siguiente tabla con toda la información solicitada:
\end{tcolorbox}

\begin{table}[h!]
    \centering
    \begin{tabularx}{\textwidth}{|X|X|X|}
        \hline
        Color del diodo LED & $V_{LED}$ & ¿Cuántos diodos LED se han quemado en el proceso? \\
        \hline
         \rule{0pt}{1cm}  & &  \\ 
         \hline
    \end{tabularx}
\end{table}

\begin{tcolorbox}[mybox, title=Ejercicio 2]
Una vez terminado el proceso de fabricación hay que realizar un documento explicativo con toda la información relevante del proyecto.

\end{tcolorbox}


\end{document}