\documentclass[a4paper,11pt]{article}

% Importar preamble de prácticas
% =============================================================================
% PREAMBLE PARA DOCUMENTOS DE PRÁCTICAS
% =============================================================================
% Este preamble está diseñado para hojas de prácticas y ejercicios
% Incluye cajas personalizadas, tablas, y formato para trabajo de laboratorio

% --- Paquetes básicos ---
\usepackage[spanish]{babel}
\usepackage[utf8]{inputenc}
\usepackage{amsmath, amssymb}
\usepackage{graphicx}
\usepackage{float}
\usepackage{multicol}
\usepackage{array}

% --- Geometría de página ---
\usepackage{geometry}
\geometry{left=2.5cm,right=2.5cm,top=2.5cm,bottom=2.5cm}

% --- Cajas y diseño ---
\usepackage{tcolorbox}
\usepackage{booktabs}

% --- Estilo para títulos ---
\usepackage{titlesec}
\titleformat{\section}{\bfseries\large}{\thesection}{1em}{}
\titleformat{\subsection}{\bfseries}{\thesubsection}{1em}{}

% --- Cabecera y pie de página ---
% NOTA: Personaliza \rhead en cada documento según la asignatura
\usepackage{fancyhdr}
\pagestyle{fancy}
\fancyhf{}
\lhead{Prácticas}
% \rhead se define en cada documento individual
\cfoot{\thepage}

% --- Estilo de caja para ejercicios (blanco y negro) ---
\tcbset{
  mybox/.style={
    colback=white,           % fondo blanco
    colframe=black,          % borde negro
    fonttitle=\bfseries,
    boxrule=0.8pt,           % grosor del borde
    arc=3mm,                 % esquinas redondeadas
    enhanced,
    sharp corners=downhill,  % estilo limpio
  }
}


\begin{document}

\begin{center}
    \Huge \textbf{Práctica 5. Circuito rectificador de media onda.} \\[0.5cm]
\end{center}

\noindent \large Fecha: \\
\large Nombre alumno:\\

\begin{tcolorbox}[mybox, title= Ejercicio 1]
Monta el circuito del ejercicio 2 en el Tinkercad. Visualiza tanto la señal de entrada y de salida del circuito.
\end{tcolorbox}

\vspace{0.5cm}

\begin{tcolorbox}[mybox, title= Ejercicio 2]
Monta el siguiente circuito de esquema en una placa protoboard y utilizando el osciloscopio realiza una comparación de la señal de entrada y la de salida.
\end{tcolorbox}
\begin{center}
    \begin{circuitikz}[american voltages, european resistors]
    
        \draw (0,0) node{} to[sV, l_={$V_{AC}=5V$}] (0,3);
    
        \draw  (0,3) to[D, l_={\small 1N4007}] (8,3);
            
        \draw (8,3) to[R, l_={$R=1\,\text{k}\Omega$}] (8,0);
    
        \draw (8,0) -- (0,0);

    \end{circuitikz}
\end{center}

    \begin{tikzpicture}[scale=1.5]
    
        % Marco de pantalla
        \draw[rounded corners=8pt, thick] (-5,-3.5) rectangle (5,3.5);
    
        % Cuadrícula fina
        \draw[step=0.5cm,gray!40] (-4.5,-3) grid (4.5,3);
    
        % Cuadrícula gruesa
        \draw[step=1cm,gray!70,thick] (-4.5,-3) grid (4.5,3);
    
        % Ejes
        \draw[->, thick] (-4.5,0) -- (4.5,0) node[right] {};
        \draw[->, thick] (0,-3) -- (0,3) node[above] {};
    
    \end{tikzpicture}







\end{document}
