\documentclass[a4paper,11pt]{article}

% Importar preamble de prácticas
% =============================================================================
% PREAMBLE PARA DOCUMENTOS DE PRÁCTICAS
% =============================================================================
% Este preamble está diseñado para hojas de prácticas y ejercicios
% Incluye cajas personalizadas, tablas, y formato para trabajo de laboratorio

% --- Paquetes básicos ---
\usepackage[spanish]{babel}
\usepackage[utf8]{inputenc}
\usepackage{amsmath, amssymb}
\usepackage{graphicx}
\usepackage{float}
\usepackage{multicol}
\usepackage{array}

% --- Geometría de página ---
\usepackage{geometry}
\geometry{left=2.5cm,right=2.5cm,top=2.5cm,bottom=2.5cm}

% --- Cajas y diseño ---
\usepackage{tcolorbox}
\usepackage{booktabs}

% --- Estilo para títulos ---
\usepackage{titlesec}
\titleformat{\section}{\bfseries\large}{\thesection}{1em}{}
\titleformat{\subsection}{\bfseries}{\thesubsection}{1em}{}

% --- Cabecera y pie de página ---
% NOTA: Personaliza \rhead en cada documento según la asignatura
\usepackage{fancyhdr}
\pagestyle{fancy}
\fancyhf{}
\lhead{Prácticas}
% \rhead se define en cada documento individual
\cfoot{\thepage}

% --- Estilo de caja para ejercicios (blanco y negro) ---
\tcbset{
  mybox/.style={
    colback=white,           % fondo blanco
    colframe=black,          % borde negro
    fonttitle=\bfseries,
    boxrule=0.8pt,           % grosor del borde
    arc=3mm,                 % esquinas redondeadas
    enhanced,
    sharp corners=downhill,  % estilo limpio
  }
}




% -------------------------
\begin{document}

\begin{center}
    \Huge \textbf{Práctica 6. Circuito rectificador de onda completa.} \\[0.5cm]
\end{center}

\noindent \large Fecha: \\
\large Nombre alumno:\\

\begin{tcolorbox}[mybox, title= Ejercicio 1]
Monta el siguiente circuito en el Tinkercad. Visualiza tanto la señal de entrada y de salida del circuito.
\end{tcolorbox}

\vspace{0.5cm}

\begin{center}
    \begin{figure}[h!]
        \centering
        \includegraphics[width=0.6\linewidth]{img/ful_wave.png}
    \end{figure}
\end{center}




\begin{tcolorbox}[mybox, title= Ejercicio 2]
Monta el siguiente circuito en el Tinkercad. Visualiza tanto la señal de entrada y de salida del circuito. Realiza diferentes pruebas con condensadores de: $C_1= 10\mu F$ y $C_2= 100\mu F$
\end{tcolorbox}

\begin{figure}[h!]
    \centering
    \includegraphics[width=0.7\linewidth]{img/full_wave_capacitor.png}
\end{figure}


\begin{tcolorbox}[mybox, title= Ejercicio 3]
Monta el circuito del ejercicio 1 en una placa protoboard y utilizando el osciloscopio realiza una comparación de la señal de entrada y la de salida.
\end{tcolorbox}

\vspace{2cm}

    \begin{tikzpicture}[scale=1.5]
    
        % Marco de pantalla
        \draw[rounded corners=8pt, thick] (-5,-3.5) rectangle (5,3.5);
    
        % Cuadrícula fina
        \draw[step=0.5cm,gray!40] (-4.5,-3) grid (4.5,3);
    
        % Cuadrícula gruesa
        \draw[step=1cm,gray!70,thick] (-4.5,-3) grid (4.5,3);
    
        % Ejes
        \draw[->, thick] (-4.5,0) -- (4.5,0) node[right] {};
        \draw[->, thick] (0,-3) -- (0,3) node[above] {};
    
    \end{tikzpicture}

\end{document}
