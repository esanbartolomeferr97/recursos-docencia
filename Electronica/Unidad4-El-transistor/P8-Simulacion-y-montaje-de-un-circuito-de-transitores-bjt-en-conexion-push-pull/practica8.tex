\documentclass[a4paper,11pt]{article}

% Importar preamble de prácticas
% =============================================================================
% PREAMBLE PARA DOCUMENTOS DE PRÁCTICAS
% =============================================================================
% Este preamble está diseñado para hojas de prácticas y ejercicios
% Incluye cajas personalizadas, tablas, y formato para trabajo de laboratorio

% --- Paquetes básicos ---
\usepackage[spanish]{babel}
\usepackage[utf8]{inputenc}
\usepackage{amsmath, amssymb}
\usepackage{graphicx}
\usepackage{float}
\usepackage{multicol}
\usepackage{array}

% --- Geometría de página ---
\usepackage{geometry}
\geometry{left=2.5cm,right=2.5cm,top=2.5cm,bottom=2.5cm}

% --- Cajas y diseño ---
\usepackage{tcolorbox}
\usepackage{booktabs}

% --- Estilo para títulos ---
\usepackage{titlesec}
\titleformat{\section}{\bfseries\large}{\thesection}{1em}{}
\titleformat{\subsection}{\bfseries}{\thesubsection}{1em}{}

% --- Cabecera y pie de página ---
% NOTA: Personaliza \rhead en cada documento según la asignatura
\usepackage{fancyhdr}
\pagestyle{fancy}
\fancyhf{}
\lhead{Prácticas}
% \rhead se define en cada documento individual
\cfoot{\thepage}

% --- Estilo de caja para ejercicios (blanco y negro) ---
\tcbset{
  mybox/.style={
    colback=white,           % fondo blanco
    colframe=black,          % borde negro
    fonttitle=\bfseries,
    boxrule=0.8pt,           % grosor del borde
    arc=3mm,                 % esquinas redondeadas
    enhanced,
    sharp corners=downhill,  % estilo limpio
  }
}


% -------------------------
\begin{document}

\begin{center}
    \Huge \textbf{Práctica 8. Simulación y montaje de un circuito de transistores  BJT en conexión \textit{push-pull}.} \\[0.5cm]
\end{center}

\noindent \large Fecha: \\
\large Nombre alumno:\\

\begin{tcolorbox}[mybox, title=Ejercicio 1]
Utiliza el software de simulación de electrónica Tinkercad y realiza el siguiente montaje. Observa que ocurre con los diodos LED cuando se gira el potenciómetro a sus extremos.
\end{tcolorbox}

\begin{figure}[h!]
    \centering
    \includegraphics[width=0.7\linewidth]{img/push-pull.png}
\end{figure}


\begin{tcolorbox}[mybox, title=Ejercicio 2]
Monta el circuito del ejercicio 1 en la placa protoboard. Varía el valor del potenciómetro (entre el máximo y el mínimo) y comprueba el estado de los diodos LED. Describe su funcionamiento detalladamente.
\end{tcolorbox}


\end{document}