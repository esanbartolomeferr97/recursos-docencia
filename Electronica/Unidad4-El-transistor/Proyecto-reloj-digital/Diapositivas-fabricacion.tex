\documentclass[aspectratio=169, 12pt]{beamer}

% Importar preamble de prácticas
% =============================================================================
% PREAMBLE PARA PRESENTACIONES BEAMER
% =============================================================================
% Este preamble está diseñado para crear presentaciones con Beamer
% Incluye paquetes comunes para circuitos, gráficos, y matemáticas

% --- Paquetes básicos ---
\usepackage{pgfplots}
\usepackage{circuitikz}[americanresistors]
\usepackage{siunitx}
\usepackage{multicol} % Añade esto en el preámbulo si no lo tienes

% --- Tema y configuración de Beamer ---
\usetheme{Madrid}
\usecolortheme{whale}
\setbeamertemplate{navigation symbols}{}
\setbeamertemplate{footline}[frame number]

% --- Ocultar fecha por defecto ---
\setbeamertemplate{date}{}

% --- Numeración del índice ---
\setbeamertemplate{section in toc}[sections numbered]
\setbeamertemplate{subsection in toc}[subsections numbered]

% --- Mostrar índice al inicio de cada sección ---
\AtBeginSection[]{
  \begin{frame}{Índice}
    \tableofcontents[
      currentsection,
      currentsubsection,
      hideothersubsections
    ]
  \end{frame}
}

% --- Mostrar índice al inicio de cadaa subsección ---
%\AtBeginSubsection[]{
%  \begin{frame}{Índice}
%    \tableofcontents[
%      currentsection,
%      currentsubsection,
%      hideothersubsections
%    ]
%  \end{frame}
%}


% --- Información diapositivas ---
\title{Fabricación de una placa de circuito impreso para un reloj digital.}
\titlegraphic{\includegraphics[width=0.25\linewidth]{img/Reloj_7segmentos_arduino_superior.png}}
\subtitle{}
\author{Enric San Bartolomé Ferri \\ \texttt{e.sanbartolomeferr@edu.gva.es}}
\setbeamertemplate{date}{} % Oculta la fecha


% -------------------------
\begin{document}
%----------------------------------------

\begin{frame}
  \titlepage
\end{frame}

%----------------------------------------
\begin{frame}{Índice}
    \tableofcontents
\end{frame}
%----------------------------------------

\section{Introducción}
\begin{frame}[t]{Introducción}
    Una placa virgen, consiste en una planca de base aislante (baquelita), que servirá de soporte, y sobre una de las capas o las dos, una fina capa de cobre.
    \begin{figure}[h!]
        \centering
        \includegraphics[width=0.45\linewidth]{img/placas_baquelita_capas.png}
    \end{figure}
\end{frame}
%----------------------------------------

\begin{frame}[t]{Introducción}
    Sobre esta capa actuaremos para hacer desaparecer todo el cobre sobrante y que queden nada más las pistas que configuran el circuito.
    \begin{figure}[h!]
        \centering
        \includegraphics[width=0.35\linewidth]{img/placa_virgen.png}
    \end{figure}
\end{frame}

%----------------------------------------
\section{Normas de diseño para una realización manual}
\begin{frame}[t]{Normas de diseño}
    Se trata de realizar un diseño lo más sencillo posible, cuanto más cortas sean las pistas y más simple la distribución, mejor resultará el diseño.\\
    No se realizarán pistas con ángulos de $90^o$, cuando sea necesario efectuar un giro en una pista se hará con ángulos de $135^o$. Si es necesario realizar una bifurcación en la pista, se hará suavizando los ángulos con sendos triángulos a cada lado.
    \begin{figure}
        \centering
        \begin{minipage}{0.45\textwidth}
            \centering
            \includegraphics[width=0.85\textwidth]{img/dibujos_pistas.png}
            
        \end{minipage}
        \hfill
        \begin{minipage}{0.45\textwidth}
            \centering
            \includegraphics[width=0.85\textwidth]{img/dibujos_pistas2.png}
            
        \end{minipage}
    \end{figure}

\end{frame}


\begin{frame}[t]{Normas de diseño}
    Entre pistas próximas y puntos de soldadura, como norma general, se dejará una distancia mínima de unos $0,8 mm$; en casos de diseños complejos, se podrá disminuir hasta $0,4 mm$.\\ En algunas ocasiones será preciso cortar una porción de ciertos puntos de soldadura para que se cumpla esta norma.
    \begin{figure}
        \centering
        \begin{minipage}{0.45\textwidth}
            \centering
            \includegraphics[width=0.8\textwidth]{img/dibujos_pistas3.png}
            
        \end{minipage}
        \hfill
        \begin{minipage}{0.45\textwidth}
            \centering
            \includegraphics[width=0.8\textwidth]{img/dibujos_pistas4.png}
            
        \end{minipage}
    \end{figure}


\end{frame}
%----------------------------------------
\section{Herramientas y materiales necesarios}
\begin{frame}[t]{Herramientas y materiales necesarios}
    \begin{columns}
        \begin{column}{0.55\textwidth}
            \begin{itemize}
                \item Agua oxigenada de 100 volúmenes
                \item Salfumán
                \item Bandeja de plástico
                \item Rotulador de tinta permanente resistente al ataque de ácido
                \item Soldador electrónico
                \item Estaño
                \item Placa virgen de circuito impreso
                \item Taladro con broca de 1mm
            \end{itemize}
        \end{column}
        \begin{column}{0.45\textwidth}
            \begin{figure}
                \centering
                \includegraphics[width=0.8\textwidth]{img/materiales_necesarios.png}
            \end{figure}
        \end{column}
    \end{columns}
\end{frame}


%----------------------------------------
\section{Diseño y construcción del circuito impreso}
\subsection{Esquemático}
\begin{frame}[t]{Diseño y construcción del circuito impreso}
    En primer lugar, hay que analizar el esquemático del circuito impreso a realizar:

    \begin{figure}[h!]
        \centering
        \includegraphics[width=0.6\linewidth]{img/reloj_digital_esquematico.pdf}
    \end{figure}
\end{frame}

%----------------------------------------
\subsection{Pistas del circuito impreso}
\begin{frame}[t]{Diseño y construcción del circuito impreso}
    \begin{figure}[h!]
        \centering
        \includegraphics[width=0.35\linewidth]{img/reloj_digital_pistas_ampliado.png}
    \end{figure}
\end{frame}


%----------------------------------------

\begin{frame}[t]{Diseño y construcción del circuito impreso}
    En primer lugar, hay que posicionar la plantilla de las pistas sobre la placa de circuito impreso y asegurarla utilizando celo. A continuación, utilizando el taladro y la broca metálica de 1mm realizar todos los agujeros siguiendo la plantilla.
    \begin{figure}
        \centering
        \begin{minipage}{0.49\textwidth}
            \centering
            \includegraphics[width=1\textwidth]{img/plantilla_y_placa_virgen.png}
            
        \end{minipage}
        \hfill
        \begin{minipage}{0.49\textwidth}
            \centering
            \includegraphics[width=1\textwidth]{img/plantilla_y_placa_virgen_agujereada.png}
            
        \end{minipage}
    \end{figure}
\end{frame}



%----------------------------------------

\begin{frame}[t]{Diseño y construcción del circuito impreso}
    Una vez realizados todos los agujeros de la plantilla, limpiar la placa de cobre dejándola libre de todo tipo de suciedad y con el rotulador de tinta permanente resistente al ácido realizar todas las pistas siguiendo los dibujos de la plantilla.
    \begin{figure}[h!]
        \centering
        \includegraphics[width=0.5\linewidth]{img/plantilla_y_placa_con_dibujos.png}
    \end{figure}

\end{frame}

%----------------------------------------

\begin{frame}[t]{Diseño y construcción del circuito impreso}
    \begin{itemize}
        \item A continuación se procede al atacado del ácido. Para ello, se realizará una mezcla de salfumán,  agua oxigenada de 100 volúmenes y agua de grifo a partes iguales.
    \end{itemize}
    \begin{alertblock}{!CUIDADO!}
        El ácido obtenido es muy corrosivo. Si no se maneja con cuidado puede provocar deterioros en la piel o la ropa, por lo que debe prestarse la máxima atención cuando se manipule. Además debe realizarse en un sitio con abundante agua y muy bien ventilado.
    \end{alertblock}
    \begin{itemize}
        \item Sitúa el ácido sobre una cubeta de plástico y se introduce la placa. Dejar actuar la mezcla dando un ligero movimiento a la cubeta observando la placa. Una vez que ha desaparecido todo el cobre, retirar la placa con cuidado y colocarla debajo de un grifo con agua abundante.
    \end{itemize}
\end{frame}


%----------------------------------------

\begin{frame}[t]{Diseño y construcción del circuito impreso}
    Cuando ya está seca la placa, se elimina la tinta que cubre el cobre, para ello, se puede utilizar disolvente o un estropajo. Una vez seca se puede depositar una fina capa de barniz protector soldable para evitar que se oxiden las pistas.
\end{frame}

%----------------------------------------

\begin{frame}[t]{Diseño y construcción del circuito impreso}
    
\end{frame}

%----------------------------------------
\section{Montaje de los componentes}
\begin{frame}{Montaje de los componentes}
    
\end{frame}

%----------------------------------------

\begin{frame}{Materiales}
    
\end{frame}



\end{document}